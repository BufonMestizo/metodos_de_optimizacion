\documentclass[12pt]{article}
	
\usepackage[margin=1in]{geometry}
\usepackage{amsmath}
\usepackage{graphics}
\usepackage{fancyhdr}				
\usepackage{graphicx}			
\usepackage{cancel}		
\usepackage[spanish]{babel}
\usepackage{hyperref}

%%%%%%%%%%%%%%%%%%%%%%
% Set up fancy header/footer
\pagestyle{fancy}
\fancyhead[LO,L]{Métodos de ptimización}
\fancyhead[CO,C]{FINESI}
\fancyhead[RO,R]{13 de enero de 2025}
\fancyfoot[LO,L]{}
\fancyfoot[CO,C]{\thepage}
\fancyfoot[RO,R]{}
\renewcommand{\headrulewidth}{0.4pt}
\renewcommand{\footrulewidth}{0.4pt}
%%%%%%%%%%%%%%%%%%%%%%

\begin{document}

\noindent \textbf{\large Universidad Nacional del Altiplano\\
Facultad de Ingeniería Estadística e Informática\\
Docente: } \large Fred Torres Cruz\\
\textbf {\large Autor :} Ronald Junior Pilco Nuñez
\begin{center}
 \section*{Trabajo Encargado - N° 001}
 \section*{Graficador de Funciones}
\end{center}

\section{Introducción}

\noindent Se desarrolló una aplicación para graficar funciones matemáticas de una variable usando Python, con Tkinter para la interfaz gráfica y Matplotlib para la visualización.

\section{Tecnologías}

\begin{itemize}
    \item \textbf{Python}: Lenguaje multiparadigma, ideal por su simplicidad y amplio ecosistema.
    \item \textbf{Tkinter}: Biblioteca de python para construir la interfaz de usuario.
    \item \textbf{Matplotlib}: Biblioteca de python para generación de gráficas personalizables.
    \item \textbf{NumPy}: Biblioteca de python para anejo eficiente de datos numéricos.
\end{itemize}

\section{Arquitectura}

\begin{itemize}
    \item \textbf{Interfaz}: Permite ingresar la función y el intervalo.
    \item \textbf{Cálculo}: Evalúa la función en el rango indicado.
    \item \textbf{Visualización}: Genera la gráfica con cuadrícula, etiquetas y límites definidos.
\end{itemize}

\section{Características}

\begin{itemize}
    \item Entrada de funciones matemáticas (ej. \( x^2 \)).
    \item Configuración del rango de la variable.
    \item Gráfica con cuadrícula, etiquetas y leyenda.
\end{itemize}

\section{Código}
\begin{figure}[h!]
    \centering
    \includegraphics[width=0.95\textwidth]{Código.png}
    \caption{Código en Python}
    \label{fig:ejemplo-grafica}
\end{figure}


\begin{figure}[h!]
    \centering
    \includegraphics[width=1\textwidth]{Graficador.png}
    \caption{Ejemplo de gráfica generada por la aplicación.}
    \label{fig:ejemplo-grafica}
\end{figure}

\section{Conclusión}

\noindent Python y sus bibliotecas permiten crear herramientas intuitivas y funcionales para la visualización matemática. Futuras mejoras podrían incluir validación más robusta y exportación de gráficas.

\section*{Github}
\href{https://github.com/BufonMestizo/metodos_de_optimizacion/blob/master/TRABAJO%20NRO%20001.py}{Repositorio Git}

\end{document}
